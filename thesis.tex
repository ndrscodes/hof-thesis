\documentclass[a4paper,final]{hofuniversityinf}
\usepackage[ngerman]{babel}
\usepackage[
    backend=biber,
    style=alphabetic,
    sorting=ynt
]{biblatex}

\title{Die Zahl 42}
\subtitle{Beweis der Allmächtigkeit}
\author{
    Andreas Schmidt\\
    Alfons-Goppel-Platz 1\\
    95028 Hof
}
\supervisor{
    Dr.\ Prof.\ Mustermann\\
    Alfons-Goppel-Platz 1\\
    95028 Hof
}
\course{Test}
\faculty{Test}
\date{\today}
\useborderspec
%\company{Company} --> this is only required if you need a blocking note.

\begin{document}
%there are two ways of defining the title, ToC and blocking note sections. Either:
%\maketitle
%\pagenumbering{roman} --> switches the page numbering to roman letters
%\blockingnote --> only if you need a blocking note
%\maketoc
%\pagenumbering{arabic}

%OR: \makehead will automatically create the head sections and switch the numbering accordingly.
%\setblocked --> call this if you need a blocking note after the title.
\makehead

%uncomment in order to create a blocking note

%creates pages like glossary, table of contents etc.
%\maketoc{}

\section{Einleitung}

\section{Weiteres Kapitel}

\printbibliography

\end{document}